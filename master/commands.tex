\newcommand{\mcN}{\mathcal{N}}

% table of contents and chapter numbering
\newcommand*\chapternumber[1]{\setcounter{chapter}{\numexpr#1 - 1\relax}}
\newcommand*\newchapter[3]{
	\chapternumber{#1}
	\addcontentsline{loc}{chapter}{#3}
	\input{#2}
}

% for important reader notes
\newcommand*\readernote[1]{\begin{footnotesize}\begin{spacing}{1.0}\begin{aside}$\star$ #1\end{aside}\end{spacing}\end{footnotesize}}
\newcommand*\asterisknote[1]{\begin{footnotesize}\begin{spacing}{1.0}$\star$ #1\end{spacing}\end{footnotesize}}

% theorem environments
\tcbuselibrary{theorems}
\tcbuselibrary{breakable}
\tcbuselibrary{skins}

% colors for boxes
\definecolor{light-gray}{gray}{0.6}
\definecolor{lightlight-gray}{gray}{0.95}
\definecolor{light-blue}{RGB}{135,206,250}
\definecolor{crimson}{RGB}{220,20,60}
\definecolor{baby-blue}{RGB}{200,230,255}

% formatting for commonly used things
% definitions
\newtcbtheorem[number within=section]{definition}{Definition}%
{theorem style=plain,colback=light-gray,colframe=white,coltitle=black,fonttitle=\bfseries,breakable,enhanced,
oversize,sharp corners, before skip=0.5\baselineskip}{def}
% derivations
\newtcbtheorem[number within=section]{derivation}{Derivation}%
{theorem style=plain,colback=lightlight-gray,colframe=white,coltitle=black,fonttitle=\bfseries,breakable,enhanced,
oversize,sharp corners, before skip=0.5\baselineskip}{der}
% examples
\newcounter{example}
\newtcbtheorem[number within=chapter, use counter=example]{example}{Example}%
{theorem style=plain,colback=baby-blue,colframe=white,coltitle=black,fonttitle=\bfseries,breakable,enhanced,
oversize,sharp corners, before skip=0.5\baselineskip}{ex}
% connection to ML cube
\newtcolorbox{mlcube}[1]{colback=white,colframe=light-blue!75,coltitle=black,fonttitle=\bfseries,breakable,enhanced,
oversize,sharp corners, before skip=0.5\baselineskip,title=ML Framework Cube: #1}
% asides
\newtcolorbox{aside}{colback=white,colframe=crimson!75,coltitle=black,fonttitle=\bfseries,breakable,enhanced,
oversize,sharp corners, before skip=0.5\baselineskip}

% chapter formatting
% \titleformat{\chapter}[display]
%   {\bfseries\slshape\huge}
%   {\flushleft \textbf{\upshape\thechapter}}
%   {-1ex}
%   {
%     \bigrule
%     \vspace{-0.8ex}
%     \huge
%   }
%   [{\vspace{-2ex}\lilrule}]